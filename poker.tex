\documentclass[utf8]{article}
\usepackage[utf8]{inputenc}
\usepackage{venturis}
\usepackage[T1]{fontenc}
\usepackage{geometry}
\geometry{papersize={12cm,15cm}}%{9.68cm,13.764cm}}
\geometry{left=0.5cm,right=0.5cm,top=0.5cm,bottom=0.5cm}
\usepackage{graphicx}
\usepackage{geometry}
\usepackage{fancyhdr}
\usepackage{etoc}
\usepackage{tasks}
\usepackage{tipa}
\usepackage{float}
\usepackage{xparse}
\usepackage{multicol}
\usepackage{enumitem}
\usepackage{amsmath, amssymb, bm}
\usepackage{mathtools}
\usepackage{physics}
\usepackage{tabularx}
\usepackage{booktabs}
\usepackage{extarrows}
\usepackage{xeCJKfntef}
\usepackage{tikz}
\usepackage[most]{tcolorbox}
\usepackage{stackengine}
\usepackage[calc]{datetime2}
\usepackage{rotating}
\usepackage{xcolor}
\usepackage{color}  
\usepackage{fontspec}
\usepackage{pifont}
\usepackage{arev}
\usepackage{multicol} 
\usepackage{xeCJK}
\setCJKfamilyfont{heiti}{黑体}
\setCJKfamilyfont{kaishu}{楷体}
\setCJKfamilyfont{wryh}{微软雅黑}
\newcommand{\heiti}{\CJKfamily{heiti}}
\newcommand{\kaishu}{\CJKfamily{kaishu}}
\newcommand{\wryh}{\CJKfamily{wryh}}
\newfontfamily\Tahoma{Tahoma}
\newfontfamily\ccs{Card Characters.ttf}
\setCJKmonofont {KaiTi}
\newlength\cardheight
\newlength\cardwidth
\setlength\cardheight{12.7cm}%0.6\textheight}%13.2-0.5=12.7
\setlength\cardwidth{9.45cm}%0.56\textwidth}
% Standard poker-size: 6.3 cm x 8.8 cm
\newcommand\card[9]{%
\begingroup\fboxsep=5mm%
\fbox{
\begin{minipage}[t][\cardheight][t]
{\dimexpr\cardwidth-2\fboxsep-2\fboxrule\relax}%
\vspace{0.2cm}
\hspace{0.5mm}\fontsize{9.48mm}{0.5mm}\scalebox{1.45}[1.65]{\Tahoma{#1}}%
\vspace{0.2cm}\newline
\fontsize{9.48mm}{0.5mm}{\scalebox{1.15}[1.3]{#2}}%
\vspace{-2.5cm}\newline
\centerline{\fontsize{9.48mm}{0.5mm}\ \ \heiti #3}
\vspace{0.2cm}\newline
\hspace{-8mm}\centerline{\Large \Tahoma #4}
\vspace{0.5cm}\newline
\centerline{\includegraphics[width=0.6\cardwidth]  {#5}}
\vspace{0.3cm}\newline
\centerline{\fontsize{4.83mm}{0.5mm}{\texttt{#6} }}
\vspace{0.1cm}\newline
\centerline{\fontsize{4.83mm}{0.5mm}{\kaishu #7}}
\vspace{-0.8cm}
\vfill{%
\rightline{\begin{sideways} \begin{sideways}\fontsize{9.48mm}{0.5mm}\scalebox{1.15}[1.15]{#2}  \end{sideways}\end{sideways}}
\vspace{0.3cm}
#8 \hfill{ \begin{sideways} \begin{sideways}\hspace{0.5mm}\fontsize{9.48mm}{0.5mm}\scalebox{1.45}[1.65]{\Tahoma{#1}}  \end{sideways}\end{sideways}}
\newline
\vspace{-0.2cm}
}
\end{minipage}
}%
\hspace{0.376mm}%
\endgroup}%


\newcommand\cardfull[9]{
\card{#1}{#2}{#3}{#4}{#5}{#6}{#7}{#8}{#9}
\clearpage
}

\newcommand{\blank}{\hrulefill}

\pagestyle{empty}

\begin{document}
\cardfull{\color{red}\ccs{3}}{{\color{red}\ccs{\{}}}{吴耀琨\ }{Yaokun Wu}{YaokunWu-removebg-preview.png}{这才是真正的数学}{ \Large{$\operatorname{det} L_{I}=f_{D}(I)$}}{wyk@sjtu.edu.cn}{1mm}
\cardfull{\color{red}\ccs{8}}{{\color{red}\ccs{\{}}}{张晓东\ }{Xiaodong Zhang}{XiaodongZhang-removebg-preview.png}{\small loxuesor, niss zongber koyer hense chegor}{ \Large{$(x)_n=\sum_{k=0}^n s(n,k)x^k$}}{xiaodong@sjtu.edu.cn}{1mm}
\cardfull{\color{red}\ccs{K}}{{\color{red}\ccs{\{}}}{石雨昂\ }{Yuang Shi}{YuangShi-removebg-preview.png}{一楼的保安亭是我的}{ \Large{$\int_{\Omega} \mathrm{d} \omega=\int_{\partial \Omega} \omega$}}{yuangshi@sjtu.edu.cn}{1mm}
\cardfull{\ccs{Q}}{{\ccs{\}}}}{李亚纯\ }{Yachun Li}{YachunLi-removebg-preview.png}{“简述二维Harnack不等式并证明”}{ \large{$u(x) \leq \frac{R^{n-2}(R+|x|)}{(R-|x|)^{n-1}} u(0)$}}{ycli@sjtu.edu.cn}{1mm}
\cardfull{\color{red}\ccs{=}}{{\color{red}\ccs{\{}}}{韩\ 东\ }{Dong Han}{DongHan-removebg-preview.png}{期末考试比期中简单}{ \Large{$\mathrm{E}X=\mathrm{E}(\mathrm{E}(X|Y))$}}{donghan@sjtu.edu.cn}{1mm}
\cardfull{\color{black}\ccs{K}}{{\color{black}\ccs{\}}}}{周颀\ }{Qi Zhou}{QiZhou-removebg-preview.png}{一楼的保安亭给石雨昂}{ \Large{$E=\frac{\pi^{\frac{d}{2}}}{2L^da^{\frac{d}{2}}}\sum\limits_{\boldsymbol{k}}|\rho(\boldsymbol{k})|^2e^{-\frac{|\boldsymbol{k}|^2}{4a}}$}}{zhouqi1729@sjtu.edu.cn}{1mm}
\cardfull{\color{red}\ccs{A}}{{\color{red}\ccs{\{}}}{王维克\ }{Weike Wang}{WeikeWang-removebg-preview.png}{关起门来捉老鼠}{ \Large{$f(\partial \Omega)=\partial \Omega \Rightarrow \exists f(x)=x$}}{wkwang@sjtu.edu.cn}{1mm}

% % This is used to test for symmetry.
% % Comment out the code "\clearpage" when using.
% \centering
% \cardfull{\ccs{Q}}{{\ccs{\}}}}{李亚纯\ }{Yachun Li}{YachunLi-removebg-preview.png}{“简述二维Harnack不等式并证明”}{ \large{$u(x) \leq \frac{R^{n-2}(R+|x|)}{(R-|x|)^{n-1}} u(0)$}}{ycli@sjtu.edu.cn}{1mm}
% \hspace{-1.25cm}
% \vspace{-13.75cm}\newline
% \begin{sideways} \begin{sideways}
%     \cardfull{\ccs{Q}}{{\ccs{\}}}}{李亚纯\ }{Yachun Li}{YachunLi-removebg-preview.png}{“简述二维Harnack不等式并证明”}{ \large{$u(x) \leq \frac{R^{n-2}(R+|x|)}{(R-|x|)^{n-1}} u(0)$}}{ycli@sjtu.edu.cn}{1mm}
% \end{sideways}\end{sideways}

\end{document}
